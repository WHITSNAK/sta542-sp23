\documentclass[11pt, letterpaper]{article}
\usepackage[utf8]{inputenc}
\usepackage[letterpaper, margin=0.5in]{geometry}
\usepackage{amsmath}
\usepackage{amssymb}
\usepackage{amsthm}
\usepackage{graphicx}
\usepackage{listings}
\usepackage[font=scriptsize]{caption}
\usepackage{subcaption}
\usepackage{xcolor}

\newtheorem{lemma}{Lemma}
\newcommand{\indep}{\perp \!\!\! \perp}

\definecolor{codegreen}{rgb}{0,0.6,0}
\definecolor{codegray}{rgb}{0.5,0.5,0.5}
\definecolor{codepurple}{rgb}{0.58,0,0.82}
\definecolor{backcolour}{rgb}{0.95,0.95,0.92}

\lstdefinestyle{mystyle}{
    backgroundcolor=\color{backcolour},   
    commentstyle=\color{codegreen},
    keywordstyle=\color{magenta},
    numberstyle=\tiny\color{codegray},
    stringstyle=\color{codepurple},
    basicstyle=\ttfamily\footnotesize,
    breakatwhitespace=false,
    texcl=true,
    mathescape=true,
    breaklines=true,                 
    captionpos=b,                    
    keepspaces=true,                 
    numbers=left,                    
    numbersep=5pt,                  
    showspaces=false,                
    showstringspaces=false,
    showtabs=false,                  
    tabsize=2
}

\lstset{style=mystyle}
\graphicspath{ {.} }
\captionsetup{justification=raggedright, singlelinecheck=false}

\author{Ryan Tang}
\title{STA 542 HW 1}
\date{February 9th 2023}

\begin{document}
\maketitle

\section{Ex 1}
We are given a MA(1) model $x_t = \theta w_{t-1} + w_t, \theta=2.3$. Assuming Gaussian noise $w_t \thicksim N(0, v)$. We have the following properties. That any trace $x_{1:n}$ follows a multivariate-normal joint with $0$ mean and tri-diagonal covariance matrix.
\begin{align*}
    x_t &\thicksim N(0, v(1+\theta^2)) \\
    x_{1:n} &\thicksim N(0, \Gamma_n) \\
    \Gamma_n &= v \begin{bmatrix}
        1+\theta^2 & \theta & 0 & \dots & 0 \\
        \theta & 1+\theta^2 & \theta & 0 & \vdots \\
        \vdots & \ddots & \ddots & \ddots & \vdots\\
        \vdots & \dots & \theta & 1+\theta^2 & \theta \\
        0 & \dots & 0 & \theta & 1+\theta^2 \\
    \end{bmatrix} \\
\end{align*}
With this, we can simply read off the ACF.
\begin{align*}
    corr(x_{t+h}, x_t) &= \frac{cov(x_{t_h}, x_t)}{v(1+\theta^2)} \\
        &= \begin{cases}
            1 & h=0 \\
            \frac{\theta}{1+\theta^2} & |h|=1 \\
            0 & |h|>0 \\
        \end{cases}
\end{align*}

To derive the PACF function, it is more involved. The definition of PACF is equalvient to $corr(x_{t+h}, x_t|x_{n/\{t+h, t\}})$ Due to normality, any conditional bi-variate $(x_{t+h}, x_t | x_{n/\{t+h, t\}})$ is also Gaussian. Hence, we can use the block matrix lemma to arrive at the conditional bivariate correlation, PACF.
\begin{align*}
    \Gamma &= \begin{bmatrix} V_1 & R \\ R^{\intercal} & V_2 \end{bmatrix} \\
    K &= \Gamma^{-1} = \begin{bmatrix} K_1 & H \\ H^{\intercal} & K_2 \end{bmatrix} \\
    (x_1 | x_2) &\thicksim N(A_1x_2, K_1^{-1}) \\
    x_1 &= [x_{t+h}, x_t]' \quad\quad x_2 = [x_{t+1}, x_{t+2}, \dots x_{t+h-1}]' \\
    A_ &= RV_2^{-1} \\
    K_1^{-1} &= V_1 - RV_2^{-1}R^{\intercal} \\
    corr(x_{t+h}, x_t|x_{n/\{t+h, t\}}) &= \phi_{hh} = - \frac{(-\theta)^h}{\sum_{i=0}^h \theta^{2i}}
\end{align*}

\section{Ex 2}
\begin{figure*}[!h]
  \centering
  \includegraphics[width=0.5\textwidth]{1.b.png}
  \captionsetup{justification=centering}
  \caption{$\theta$ Posterior Joint}
\end{figure*}

\end{document}